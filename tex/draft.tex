\documentclass[12pt,oneside,reqno]{amsart}
\usepackage[utf8]{inputenc}        % UTF8 fonts
\usepackage[T1]{fontenc}           % Fix font encoding issues with accented chars
\usepackage{fouriernc}
\usepackage[a4paper,portrait,left=1.5cm,outer=3.5cm,headheight=15pt,bottom=3cm,top=3cm]{geometry}
\usepackage{hyperref}
\usepackage{amsmath,amssymb,amsfonts,amsthm}
\usepackage{graphicx}
\usepackage{xcolor}
\usepackage[color]{showkeys}
\definecolor{refkey}{rgb}{0.3,0.5,0.3}
\definecolor{labelkey}{rgb}{0.3,0.5,0.3}
\definecolor{todo}{rgb}{0.6,0.9,0.5}
\definecolor{refkey}{rgb}{0.3,0.5,0.3}
\definecolor{labelkey}{rgb}{0.3,0.5,0.3}
\definecolor{todo}{rgb}{0.6,0.9,0.5}
\usepackage[textwidth=3cm,textsize=tiny,color=todo]{todonotes}


% fix listoftodos for amaart
\makeatletter
\providecommand\@dotsep{5}
\def\listtodoname{List of Todos}
\def\listoftodos{\@starttoc{tdo}\listtodoname}
\makeatother

\title{Solving MPB}
\begin{document}
\maketitle

\tableofcontents{}
\listoftodos{}

\section{Introduction}
We describe the modified Poisson-Boltzmann system as described in  \cite{MuellerFuhrmannLandstorfer2020} and
adapt it to the symmetric cell calculations of interest.

\section{The model}

Let $\Omega = (0,L)$.

\subsection{Poisson equation}

\begin{equation}\label{eq:Poisson}
  \begin{aligned}
    -\nabla \varepsilon \nabla \phi & = \rho(\phi,p) \; \text{in} \Omega \\
  \end{aligned}
\end{equation}

We set two variants of boundary conditions:
\begin{itemize}
  \item symmetric cell with ideally polarizable electroded and fixed charges:
        \begin{equation}
          \label{eq:symmcell}
          \begin{aligned}
            \varepsilon\nabla\phi \cdot \vec n |_{x=0} & = q  \\
            \varepsilon\nabla\phi \cdot \vec n |_{x=L} & = -q
          \end{aligned}
        \end{equation}

  \item applied potential difference for half cell with ideally polarizable electrode and bulk interface:
        \begin{equation}
          \label{eq:halfcell}
          \begin{aligned}
            \varepsilon\nabla\phi \cdot \vec n |_{x=0} & = U      \\
            \varepsilon\nabla\phi \cdot \vec n |_{x=L} & = \phi^E
          \end{aligned}
        \end{equation}
\end{itemize}


Let $\alpha=0\dots N$ be the species index, with the assumption that $\alpha=0$ denotes
the electroneutral solvent.
Then we have for the mole fractions
\begin{align}\label{eq:y}
  y_\alpha(\phi,p)= y_\alpha^E\exp\left( - \frac{z_\alpha e (\phi-\phi^E) + v_\alpha(p-p^E)}{k_BT}\right)
\end{align}
and can calculate the space charge density as
\begin{align}\label{eq:rho}
  \rho(\phi,p) = e\sum_{\alpha=1}^n z_\alpha n_\alpha = e\sum_{\alpha=1}^N z_\alpha \frac{y_\alpha}{\sum_{\alpha=0}^N v_\alpha y_\alpha}.
\end{align}
\medskip

\begin{tabular}{|r|l|}
  \hline
  constant $k_B$                                    & Boltzmann constant                       \\
  constant $e$                                      & elementary charge                        \\
  given $T$                                         & temperature  (often $T=298.15K$          \\
  given $p^E$                                       & reference pressure (usually $p^E=0$)     \\
  given $\phi^E$                                    & reference potential (usually $\phi^E=0$) \\
  given $\kappa_\alpha$                             & solvation numbers ($\kappa_0=0$)         \\
  given $v_\alpha^u$                                & unsolvated molecular volumes             \\
  given $z_\alpha$                                  & charge numbers ($z_0=0$)                 \\
  given $q$ or $U$                                  & (surface) charge or applied potential    \\
  given ($\alpha=1\dots N$) $n_\alpha^E$            & reference number densities               \\
  $v_\alpha= v_\alpha^u+\kappa v_0$                 & solvated molecular volumes               \\
  $y_\alpha^E$                                      & reference mole fractions                 \\
  $n=\frac{1}{\sum_{\alpha=0}^N v_\alpha y_\alpha}$ & mixture number density                   \\
  $n_\alpha = y_\alpha n$                           & species number density                   \\
  $\phi$                                            & electrostatic potential                  \\
  $p$                                               & pressure                                 \\
  \hline
\end{tabular}
\bigskip

For the half cell case, the reference values $\ast^E$ are given at the bulk boundary $x=L$.
From a given molarity $M$ one obtains the bulk number densities $n_\alpha^E$ which
must fulfilll electroneutrality $\sum_{\alpha=1}^N n_\alpha^E =0$. This then allows to calculate
the solvent bulk number density $n_0^E=\frac{1}{v_0} - \sum_{\alpha=1}^N v_\alpha n_\alpha^E$.

For the symmetric cell case, these data can be seen as the values at $x=L/2$ for a symmetric
binary electrolyte.
\todo{How these can be interpreted in the general case?}


\subsection{Pressure calculation}


It remains to solve for the pressure.

There are two ways to solve for the pressure.

\subsubsection{Second order pressure PDE (``Pressure Poisson'')}

First, one can use the momentum balance in mechanical equilibrium
\begin{align}
  \label{eq:momentumbalance}
  \nabla p = -q\nabla \phi
\end{align}

for the pressure derived as part of the generalized Nernst-Planck-Poisson
system \cite{dreyer2013overcoming} and take the divergence as suggested in \cite{Fuhrmann2015xCPC}, arriving at
a second order PDE
\begin{equation}\label{eq:pressurePDE}
  \begin{aligned}
    - \Delta p & = q\nabla\phi  \; \text{in} \Omega.
  \end{aligned}
\end{equation}
The boundary conditions are:

\begin{itemize}
  \item Symmetric cell:
        \begin{equation}\label{eq:symmcell}
          \begin{aligned}
            \nabla p - q\nabla\phi\cdot \vec n|_{x=0,L} & =0   \\
            p|_{x=L/2}                                  & =p^E
          \end{aligned}
        \end{equation}
  \item Half cell:
        \begin{equation}\label{eq:phalfcell}
          \begin{aligned}
            \nabla p - q\nabla\phi\cdot \vec n|_{x=0} & =0   \\
            p|_{x=L}                                  & =p^E
          \end{aligned}
        \end{equation}
\end{itemize}


\subsubsection{Algebraic pressure equation}
As an alternative, one can use the approach described in \cite{MuellerFuhrmannLandstorfer2020},
and instead of the pressure PDE solve an algebraic system
\begin{align}
  \label{eq:pressureAE}
  \sum_{\alpha=0}^N y_\alpha(\phi,p)=1
\end{align}
or (as it occurred numerically to be more benign)

\begin{align}
  \label{eq:pressureAElog}
  \log \sum_{\alpha=0}^N y_\alpha(\phi,p)=0.
\end{align}

\subsubsection{Discussion}
Numerical examples show that both approaches give the same results.
\begin{itemize}
  \item  In the papers leading to \cite{MuellerFuhrmannLandstorfer2020}, \eqref{eq:pressureAE} has
        been derived using \eqref{eq:momentumbalance}. It would be interesting to see if
        one can show their equivalence within the context of this draft. \todo{Pressure equations equivalence}
  \item The algebraic condition  \eqref{eq:pressureAE} is only valid in equilibrium. For transient
        problems, one needs to resort to \eqref{eq:momentumbalance} or \eqref{eq:pressurePDE}.
  \item If all molecular volumes are equal (i.e. $\kappa_\alpha=0, v_\alpha=v_0$ for $ \alpha=1\dots n$,
        the equations decouple. It is interesting to see the behavior of the pressure here.
\end{itemize}

\section{Computations}

\subsection{Half cell problem}
This is the classical problem leading to the calculation of double layer capacitances.
From the space charge density $\rho(\phi, p)$ for given solution with applied potential $U$,
one calculates the double layer charge
\begin{align}
  \label{eq:spacecharge}
  Q_{dl}=Q_{dl}(U)=\int_\Omega \rho\; d\omega.
\end{align}
(This formula takes into account that there is no boundary layer in the system). Repeated calculations
for different applied potentials $U$ allow to estimate the space charge capacitance
\begin{align}
  \label{eq:dlcap}
  C_{dl}(U) = \frac{dQ}{dU} \approx  \frac{Q(U+\delta U)-Q(U)}{\delta U}.
\end{align}
\todo{I think we also should write code for this as part of verification}

\subsection{Symmetric cell problem}
This problem appears to be less investigated.\todo{Do some literature research here}
However, we can hypothesize some of its properties:
\begin{itemize}
  \item As we apply opposite charges at the interfaces, global electroneutrality  enforces
        \begin{align}
          \label{eq:globneutral}
          \int_\Omega \rho\; d\omega  =0
        \end{align}
  \item For this problem, we can define $U=\phi(0)-\phi(L)$, and we can calculate the values
        $U(q)$. Does inversion of this relationship provide similar double layer capacitance curves ?
        \todo{$C_{dl}$ for symmetric case}
  \item The given input data (fixing $n_\alpha^E$) do not correspond to the molecular simulation
        experiment to be compared to. IMHO, we need to fix the number of ions in the domain
        $$n_\alpha^{\Omega} = \int_\Omega n_\alpha d\omega.$$\todo{Develop an efficient
          algorithm for this calculation using AD}

        When working with the transient problem, one could just fix an initial concentration
        value and solve the evolution after applying the charge (very much like in the molecular
        simulation experiment). \todo{One should be able to show the limit mathematically, including
          free energy decay and dissipation. Also on the level of discretizations}
\end{itemize}



\section{Generalizations, alternative formulations}
\begin{itemize}
  \item We can formulate everything in molar concentrations and activity coefficients \todo{Activity coefficients?}
  \item The full problem also takes into account molar masses. \todo{Do we need to consider molar masses ?}
  \item Field dependent dielectric permittivity is rather straightforward here
        \cite{landstorfer2022thermodynamic}
        \todo{Add field dependedent   $\varepsilon$}
  \item Understand influence of concentration dependent permittivity
        \cite{landstorfer2022thermodynamic}
        \todo{Investigate concentration dependent   $\varepsilon$}
  \item Investigate ionic interaction terms, see e.g. \cite{bazant2011double}, \cite{SchammerPhD2023}
        \todo{Discuss oscillations}
\end{itemize}

\section{API}


\bibliographystyle{alpha}
\bibliography{draft}
\end{document}
